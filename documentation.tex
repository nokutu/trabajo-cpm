% !TeX spellcheck = en_US
\documentclass{article}
\usepackage[utf8]{inputenc}
\usepackage{indentfirst}


\title{Documentation}
\author{Jorge López Fueyo}

\setlength{\parskip}{1em}


\begin{document}
   \pagenumbering{gobble}
   \maketitle
   \newpage
   \pagenumbering{arabic}
     
   \section{Introduction}
   
   In this work, we have been developing a desktop application that will allow the user to book and pay for a Cruise.

   The application is suppose to read the information from a set of files and, at the end of the process, store the result in a text file, including all the necessary booking information, such us the user contact data and the cruise data.
   
   \section{Prototype}
   Our team's prototype\footnote{https://moqups.com/jorgelopezfueyo@gmail.com/nN2oP7pI} did not change much between the initial design and the final one, just an issue related with cash payment.
   
   The prototype consisted on 5 screens:
   \begin{itemize}
	   	\item Initial panel: showing sponsored offers and a search bar.
	   	\item Search panel: shows the results of the search. It also hast a search bar at the top.
	   	\item Cruise panel: shows the information of the selected cruise, e.g. description or pictures. There is a a panel on the right allowing to book it.
	   	\item Info panel: ask for the needed information such as people per cabin, extras or personal details. Allows to add the cruise to the cart and continue buying or just go ahead and pay.
	   	\item Payment panel: allows the user to choose between various payment methods in order to pay the order. 
   \end{itemize}
   
   \section{Application}
   \subsection{Design}
   The application was designed without the use of Window Builder, as, in my opinion, even though it makes things more visual, the code it creates is quite "spaghetti" and not very well structured.
   
   I have been using mainly 2 different layouts. \textbf{BorderLayout} in all the tabs of the \textbf{CardLayout} and in the JDialogs and \textbf{MigLayout}\footnote{http://miglayout.com/}, which is an improved  version of \textbf{GridBagLayout} that allows to input constraints as a String containing commands.
   
   The application design consists of a \textbf{MainFrame} instance, whose content pane has a \textbf{CardLayout}. It contains:
   \begin{itemize}
	   \item \textbf{InitialPanel}: The first panel you see when the application is opened. It just contains the application's logo and a search bar.
   	   \item \textbf{SearchPanel}: Displays the search results and allows the user to rerun searches.
	   \item \textbf{CruisePanel}: Displays the information about the selected cruise. A \textbf{BookPanel} is shown on the right size, that allows to select the type of cabin, date, extras and people.
   	   \item \textbf{PassengerInfoPanel}: In this panel the user is asked about the name and age of all the passengers in the cabin. It also checks for incompatibilities such as no minor in a room with a supplementary beds of minor in a non minor cruise.
   	   \item \textbf{PaymentPanel}: Displays the final price of the booking. A bill, which contains all the information required, is shown in the middle of the screen. From here the user can open a new dialog in order to enter the payment information.
   	   \item \textbf{FinalPanel}: Displays the bill again and allows the user to export it into a local directory.
   \end{itemize}   
   
   \subsection{Logic}
   \begin{itemize}
	   	\item \textbf{Database}: The main part of the application's logic is related with the this class class. It stores all ships, zones, cruises, cities and users of the application.
	   	\item \textbf{Preferences}: When the application is running, the preferences, such as the list of users registered or the personal settings are stored inside a HashMap in this class. Whenever the application is closed, they are saved into a text file located into the user directory, inside a folder called ".cruises". This file is then loaded when the application is started again.
	   	\item \textbf{I18n}: Stores the used internationalization methods:
		\begin{itemize}
			\item \textbf{tr(String)}: just translates the given key.
			\item \textbf{trn(String, int)}: translates the given key depending on the integer. If it is different from one, it will return the plural and the singular otherwise.
			\item \textbf{trc(String, Object[])}: translates the given key substituting the given Object inside.
		\end{itemize}
   \end{itemize}

\end{document}